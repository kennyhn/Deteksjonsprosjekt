\begin{abstract}
In this project a detector for a binary hypothesis problem is derived. Its main task is to determine if there are primary users, i.e paying customers, in part of the spectrum in a cognitive radio system. If not, the secondary users can utilize the spectrum when it is not in use by paying customers, thus taking increasing use of the radio system. Using knowledge about the statistical distributions to the signals in the network which in addition is drowned in noise, a detector with desired performance properties is derived.
The exact detector is derived for this specific problem using the known distribution and properties of the data.\\
Further, an approximate detector is calculated, which can be used to easily determine the decision rule, without needing to know all the details of the signals distribution. Another fine property of this approximate detector is that the amount of samples needed for the detector to perform well can be easily found because of its known properties.\\
Starting with the signals which is proved to be of a complex Gaussian distribution, a detector with the likelihood ratio test is derived. This detector uses a test statistic with another distribution which is the square of a complex Gaussian which belongs to $\chi^2$ distribution which essentially is a special case of the Gamma distribution. And as it turns out the detector can be approximated as a normal Gaussian variable because it utilizes the central limit theorem. This turns out to be good approximate, and much easier to handle because of its many known properties.
The results of this detector shows that it manages to correctly detect in $98\%$ of the cases with as few as 9 samples for each time step.\\

\end{abstract}