\section{Introduction}\label{sec:intro}
Detection problems occurs in many places in everyday life, in your computer you have electrical signals which the CPU has to detect as 1 or 0, or in other words a current is present or not. Another example is in air defense where the military has radars to detect if there is an hostile air action present or not. In this project we are going to look at a cognitive radio system with primary users (PU) and secondary users (SU), where the SUs are going to detect if there are PUs in the network so that they can decide if they can utilize the frequency spectrum without interfering with the quality of service (QoS) of the PUs.\\
This problem where the SUs use the frequency spectrum in an opportunistic manner whenever there are IDLE PUs is interesting since in wireless communication systems, spectrums are a scarce resource that service providers pay a substantial amount of money to the government in order to license the spectrum. This cost is covered by the customers in their monthly mobile subscription cost. Paying customers demands a certain QoS, which in the mentioned case is the PUs, and any interference appearing on the communication channel should be kept at a minimum to deliver the promised QoS.\\
This project starts by introducing the theoretical background necessary to understand and solve this problem in section \ref{sec:theory}. Then the tasks that needs to be solved for this problem is in section \ref{sec:task} followed by the implementation and results in section \ref{sec:results} then finally it is all wrapped up in \ref{sec:conclusion}.\\

Here we should write about
\begin{enumerate}[i]
	\item The goals/motivation of the course/project.
	\item Why is your task of general interest to society? etc\dots
	\item How the report is organized
	\begin{itemize}
		\item In chapter x the theory is described
		\item in chapter y the implementation is described
		\item \dots
		\item and finally the conclusion is given in chapter z
	\end{itemize}
\end{enumerate}
