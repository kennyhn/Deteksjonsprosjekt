\section{The tasks}\label{sec:task}
The main problem that is going to be solved in this project is the detection problem
\begin{align}
\begin{split}
	H_0: x(n) & = w(n), n = 0, 1, \dots, N-1\\
	H_1: x(n) & = s(n) + w(n), n = 0, 1, \dots, N-1
\end{split}
\end{align}\label{eq:detection_problem}
where $s(n)$ is a waveform sequence of the PU and $w(n)$ is an additive white complex gaussian noise.
To construct the sequence $s(n)$, that is transmitted over the wireless channel, the modulation method orthogonal frequency-division multiplexing (OFDM) is used. Each information symbol $S(k)$, $k=0,1,\dots,N-1$ is allocated on each N carrier frequencies. The unique time-domain signal $s(n)$ corresponding to the sampled spectrum is obtained by inverse discrete fourier transform. Thus is the PUs time-domain signal given by:
\begin{equation}
	s(n) = \frac{1}{\sqrt{N}}\sum_{k=0}^{N-1}S(k)e^{\frac{j2\pi nk}{N}}, n = 0,1,\dots,N-1
\end{equation}
which we notice is a complex-valued quantity.

\subsection{Task 1: Model building}
Here the data $x[n]$ is generated, which can be used in our analysis to obtain a suitable detector. To generate $x[n]$ we need do to verify that the complex-valued time-domain OFDM signal sequence $s(n) = s_R(n)+js_I(n)$ is independent and identically distributed, in addition to verify that it is accurately modelled with a complex gaussian distribution.\\
Here the datasets T1 are used as the information symbol sent over. The two datasets contains samples that are taken from a standard normal Gaussian distribution and binary phase shift keying respectively.

\subsection{Task 2: One-sample detector}
In this task only a single sample is used to generate the NP-detector. That is there needs to be done calculations to retrieve the decision rule that decides for when to choose null hypothesis over the alternative hypothesis and vice versa.

\subsection{Task 3: Performance of the one-sample detector}
Here the datasets T3 are given and are to be used to verify that
\begin{align}
	H_0 &: \frac{2x_R^2(0)}{\sigma_w^2}+\frac{2x_I^2(0)}{\sigma_w^2}=\frac{2|x(0)|^2}{\sigma_w^2}\label{eq:chi_sq_h0}\\
	H_1 &: \frac{2x_R^2(0)}{\sigma_w^2+\sigma_s^2}+\frac{2x_I^2(0)}{\sigma_w^2+\sigma_s^2}=\frac{2|x(0)|^2}{\sigma_w^2+\sigma_s^2}\label{eq:chi_sq_h1}
\end{align}
are $\chi^2$-distributed with 2 degrees of freedom. Using this it can be used to show that $\widetilde{x} = \frac{2|x(0)|^2}{\sigma^2}$ has the point-distribution function $f(\widetilde{x}) = \frac{1}{2}e^{-\frac{\widetilde{x}}{2}}$ for $\widetilde{x}>0$. This can then be used to calculate the probability for false alarm and the probability for detection for this one-sample detector.

\subsection{Task 4: General NP detector}
In this task the one-sample detector derived in the previous tasks are expanded so it can be used when there are $K>1$ samples. Since $|x(n)|^2$ is a sum of two standard normal gaussian square, then in the case where there are multiple samples, the $\chi^2$-distribution has $2K$ degrees of freedom which can be used to derive the threshold $\lambda'$ that maximizes $P_D$ and ensure that $P_{FA}<\alpha$.

\todo{Task 5-8 needs to be rewritten/reformulated}
\subsection{Task 5: Performance of the general NP detector}
In this task the distribution obtained in task 4 is used to plot the reciever operating characteristics.

\subsection{Task 6: Approximate performance of the general NP detector}
Here the central limit theorem is used to approximate the test statistic. The PDF of the test statistic is calculated, and thus can $p_D$ and $p_{FA}$ be plotted as a function of the threshold in this where the gaussian is used instead of the pdf from the gamma distribution.

\subsection{Task 7: Complexity of the detector}
Use approximation in task 6 and find an expression to compute the number of samples required to attain a given $P_{FA}$ and $P_D$.

\subsection{Task 8: Numerical experiments in PU detection}
Given a dataset of numerical experiment data, here the NP detector is applied on the dataset to decide wether a PU is present or not. Then from the results a discussion if it is reasonable or not is done.