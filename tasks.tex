\section{The tasks}\label{sec:task}
The main problem that is going to be solved in this project is the detection problem
\begin{align}
\begin{split}
	H_0: x(n) & = w(n), n = 0, 1, \dots, N-1\\
	H_1: x(n) & = s(n) + w(n), n = 0, 1, \dots, N-1
\end{split}
\end{align}\label{eq:detection_problem}
where $s(n)$ is a waveform sequence of the PU and $w(n)$ is an additive white complex gaussian noise.
To construct the sequence $s(n)$, that is transmitted over the wireless channel, the modulation method orthogonal frequency-division multiplexing (OFDM) is used. Each information symbol $S(k)$, $k=0,1,\dots,N-1$ is allocated on each N carrier frequencies. The unique time-domain signal $s(n)$ corresponding to the sampled spectrum is obtained by inverse discrete fourier transform. Thus is the PUs time-domain signal given by:
\begin{equation}
	s(n) = \frac{1}{\sqrt{N}}\sum_{k=0}^{N-1}S(k)e^{\frac{j2\pi nk}{N}}, n = 0,1,\dots,N-1
\end{equation}
which we notice is a complex-valued quantity.
\subsection{Task 1: Model building}
Here the data $x[n]$ is generated, which can be used in our analysis to obtain a suitable detector. To generate $x[n]$ we need do to verify that the complex-valued time-domain OFDM signal sequence $s(n) = s_R(n)+js_I(n)$ is independent and identically distributed, in addition to verify that it is accurately modelled with a complex gaussian distribution.\\
Here the datasets T1 are used as the information symbol sent over. The two datasets contains samples that are taken from a standard normal Gaussian distribution and binary phase shift keying respectively.
\subsection{Task 2:One-sample detector}
In this task only a single sample is used to generate the NP-detector. That is there needs to be done calculations to retrieve the decision rule that decides for when to choose null hypothesis over the alternative hypothesis and vice versa.
\subsection{Task 3: Performance of the one-sample detector}
Here the datasets T3 are given and are to be used to verify that
\begin{align}
	H_0 &: \frac{2x_R^2(0)}{\sigma_w^2}+\frac{2x_I^2(0)}{\sigma_w^2}\nonumber\\
	H_1 &: \frac{2x_R^2(0)}{\sigma_w^2+\sigma_s^2}+\frac{2x_I^2(0)}{\sigma_w^2+\sigma_s^2}\nonumber
\end{align}
are $\chi^2$-distributed with 2 degrees of freedom. \todo{Hva slags relevans har det at denne er kji-kvadratfordelt?}\todo{Har dette noe å si for teststyrken? jf. https://online.stat.psu.edu/stat800/node/66/} Then the probability of detection and false alarm are found with the one-sample-detector found in previous task.

\begin{enumerate}[i]
	\item Give a short description of all tasks and why are we doing these tasks
	\item Describe the task
\end{enumerate}